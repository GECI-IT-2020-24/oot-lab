\documentclass[12pt,a4paper]{article}
\usepackage{geometry}
\usepackage{enumitem}
\usepackage{amsmath}
\usepackage{minted}
\usepackage{xcolor}
\usepackage{fontspec}
\geometry{a4paper,
top=6mm,
left=12mm
}
\setmainfont{JetBrainsMono NF}[
    Contextuals=Alternate
]
% \definecolor{paperclrbg}{HTML}{BCBCBC}
\begin{document}
\setcounter{tocdepth}{1}
\tableofcontents
\clearpage
%Experiment 1 : 
\section{Object Oriented Programming Basics: Banking Application}
\vspace*{1mm}
% \textbf{Qn.1} Define a class to represent a bank account.Include the following members :
% \begin{minipage}[t]{0.6\textwidth}
%     \begin{enumerate}[nosep]
%         \item Name of depositor
%         \item Account Number
%         \item Account type
%         \item Account balance
%     \end{enumerate}
% \end{minipage}
% \vspace*{1mm}
\textbf{
    \inputminted[firstline=9,style=emacs]{java}{./programs/exp1.java}
}
\clearpage
%Experiment 2 : 
\section{Object Oriented Programming Basics: Student Mark Calculation}
\vspace*{1mm}
% \textbf{Qn.2} Create a class Student with instance variables :
% % \vspace*{2mm}
% \begin{enumerate}[nosep]
%     \item Name
%     \item Roll No
%     \item Mark1,Mark2,Mark3
% \end{enumerate}
% \vspace*{4mm}
% \ and methods to :
% \vspace*{4mm}
% \begin{enumerate}[nosep]
%     \item Calculate Average Mark
%     \item Display name, Rollno and Average Marks
% \end{enumerate}
% \vspace*{1mm}
\textbf{
    \inputminted[firstline=12,style=emacs]{java}{./programs/exp2.java}
}
\clearpage
%Experiment 3 :
\section{Object Oriented Programming Basics: KSEB Billing}
\vspace*{1mm}
% \textbf{Qn.3} write a class KsebConsumer with instance variables Consumer No,Consumer Name,No of units consumed
% and method CalculateCharge() to display the charge with name.
% Electricity board charges the following rates to
% domestic users to discourage large consumption of energy.
% \begin{enumerate}[nosep]
%     \item for the first \textbf{100} units :  \textbf{60} Paisa/unit
%     \item for the next  \textbf{200} units :  \textbf{80} Paisa/Unit
%     \item Beyond \textbf{300} units:  \textbf{90} Paise/Unit
% \end{enumerate}
% \vspace*{1mm}
% \textbf{All users are charged with a minimum of Rs.50.}
% if the total amount is more
% than rs 300 then an additional surcharge of Rs.15 is added
% \vspace*{1mm}
\textbf{
    \inputminted[firstline=17,style=emacs]{java}{./programs/exp3.java}
}
\clearpage
%Experiment 4 : 
\section{Method Overloading:Area Calculation}
\vspace*{1mm}
% \textbf{Qn.4} Develop a java program to calculate the area of different shapes namely
% circle,rectangle \& triangle using the concepts of method overloading.
% \vspace*{10mm}
\textbf{
    \inputminted[firstline=6,style=emacs]{java}{./programs/exp4.java}
}
\clearpage
%Experiment 5 : 
\section{Inheritance : Exmployee Details}
\vspace*{1mm}
% \textbf{Qn.5} Design a class \textbf{Person} and add Subclass \textbf{Employee}.
% Employee has two subclass \textbf{Faculty} ,\textbf{Staff}
% Person has : name,address, phone no , email id
% Student has Status \textit{( Junior , Senior , SuperSenior)}
% Define the status as a constant and employee has an office ,salary and date
% Faculty has office hours and Rank  \textit{(AssisstantProf,Associate Prof,Prof)}
% Staff member has a Title \textit{(Assistant , Officer,Manager)}
% Write a test program that creates a person, student , employee, faculty and staff \& invoke their toString method.
% \vspace*{1mm}
\textbf{
    \inputminted[firstline=12,style=emacs]{java}{./programs/exp5.java}
}
\clearpage
%Experiment 6 : 
\section{Inheritance : Student Details}
\vspace*{1mm}
% \textbf{Qn.6} Develop a java program to read and print student data using inheritance
% Class \textbf{Person} \textit{(name,age,gender)}
% Class \textbf{Student} inherits Person \textit{(mark1,mark2,mark3)}
% Compute the total mark and grade.
% \vspace*{1mm}
\textbf{
    \inputminted[firstline=8,style=emacs]{java}{./programs/exp6.java}
}
\clearpage
%Experiment 7 : 
\section{Recursive and Non-Recursive functions : Fibonacci series}
\vspace*{1mm}
% \textbf{Qn.7} Develop a java program which uses both recursive and non-recursive function
% to find the \textbf{$n^{\text{th}}$} value of fibonacci series.
% \vspace*{1mm}
\textbf{
    \inputminted[firstline=6,style=emacs]{java}{./programs/exp7.java}
}
\clearpage
%Experiment 8 : 
\section{Multithreading : Test Printing}
\vspace*{1mm}
% \textbf{Qn.8} Develop a java application that executes two threads.One thread displays \textbf{"Hello"}
% in every 1000 ms and other displays \textbf{"World"} in every 3000ms.
% Create threads by extending the thread class.
% \vspace*{1mm}
\textbf{
    \inputminted[firstline=8,style=emacs]{java}{./programs/exp8.java}
}
\clearpage
%Experiment 9 : 
\section{Multithreading : Number Generation}
\vspace*{1mm}
% \textbf{Qn.9} Develop a java program that implements a mulithreaded program that implements 3 threads.
% \begin{itemize}
%     \item first thread generates a random number for every 1 second
%     \item if the value is even second thread computes the square of the number and prints
%     \item if the value is odd the third thread will print the values of cube of the number .
% \end{itemize}
% \vspace*{1mm}
\textbf{
    \inputminted[firstline=9,style=emacs]{java}{./programs/exp9.java}
}
\clearpage
%Experiment 10 : 
\section{Exception Handling : Stack implementation}
\vspace*{1mm}
% \textbf{Qn.10} Develop a java program that implements a \textbf{Stack} that raises Exceptions
% in case of error and with \textit{push,pop \& display} functionalities.
% \vspace*{1mm}
\textbf{
    \inputminted[firstline=7,style=emacs]{java}{./programs/exp10.java}
}
\clearpage
%Experiment 11 : 
\section{AWT : Traffic Light Control}
\vspace*{1mm}
% \textbf{Qn.11} Develop a Java program that stimulates a traffic light using AWT.
% The program let's the user select one of the 3 lights \textbf{red,yellow,green} with radio button,
% on selecting a button an appropriate message with stop,ready or go
% should appear above the button in a selected colour,
% initially there is no message shown.
% \vspace*{1mm}
\textbf{
    \inputminted[firstline=9,style=emacs]{java}{./programs/exp11.java}
}
\end{document}
